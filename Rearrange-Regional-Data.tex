\documentclass[]{article}
\usepackage{lmodern}
\usepackage{amssymb,amsmath}
\usepackage{ifxetex,ifluatex}
\usepackage{fixltx2e} % provides \textsubscript
\ifnum 0\ifxetex 1\fi\ifluatex 1\fi=0 % if pdftex
  \usepackage[T1]{fontenc}
  \usepackage[utf8]{inputenc}
\else % if luatex or xelatex
  \ifxetex
    \usepackage{mathspec}
  \else
    \usepackage{fontspec}
  \fi
  \defaultfontfeatures{Ligatures=TeX,Scale=MatchLowercase}
\fi
% use upquote if available, for straight quotes in verbatim environments
\IfFileExists{upquote.sty}{\usepackage{upquote}}{}
% use microtype if available
\IfFileExists{microtype.sty}{%
\usepackage{microtype}
\UseMicrotypeSet[protrusion]{basicmath} % disable protrusion for tt fonts
}{}
\usepackage[margin=1in]{geometry}
\usepackage{hyperref}
\hypersetup{unicode=true,
            pdftitle={Rearranging Regional Data},
            pdfauthor={Daniel Antal, CFA},
            pdfborder={0 0 0},
            breaklinks=true}
\urlstyle{same}  % don't use monospace font for urls
\usepackage{color}
\usepackage{fancyvrb}
\newcommand{\VerbBar}{|}
\newcommand{\VERB}{\Verb[commandchars=\\\{\}]}
\DefineVerbatimEnvironment{Highlighting}{Verbatim}{commandchars=\\\{\}}
% Add ',fontsize=\small' for more characters per line
\usepackage{framed}
\definecolor{shadecolor}{RGB}{248,248,248}
\newenvironment{Shaded}{\begin{snugshade}}{\end{snugshade}}
\newcommand{\AlertTok}[1]{\textcolor[rgb]{0.94,0.16,0.16}{#1}}
\newcommand{\AnnotationTok}[1]{\textcolor[rgb]{0.56,0.35,0.01}{\textbf{\textit{#1}}}}
\newcommand{\AttributeTok}[1]{\textcolor[rgb]{0.77,0.63,0.00}{#1}}
\newcommand{\BaseNTok}[1]{\textcolor[rgb]{0.00,0.00,0.81}{#1}}
\newcommand{\BuiltInTok}[1]{#1}
\newcommand{\CharTok}[1]{\textcolor[rgb]{0.31,0.60,0.02}{#1}}
\newcommand{\CommentTok}[1]{\textcolor[rgb]{0.56,0.35,0.01}{\textit{#1}}}
\newcommand{\CommentVarTok}[1]{\textcolor[rgb]{0.56,0.35,0.01}{\textbf{\textit{#1}}}}
\newcommand{\ConstantTok}[1]{\textcolor[rgb]{0.00,0.00,0.00}{#1}}
\newcommand{\ControlFlowTok}[1]{\textcolor[rgb]{0.13,0.29,0.53}{\textbf{#1}}}
\newcommand{\DataTypeTok}[1]{\textcolor[rgb]{0.13,0.29,0.53}{#1}}
\newcommand{\DecValTok}[1]{\textcolor[rgb]{0.00,0.00,0.81}{#1}}
\newcommand{\DocumentationTok}[1]{\textcolor[rgb]{0.56,0.35,0.01}{\textbf{\textit{#1}}}}
\newcommand{\ErrorTok}[1]{\textcolor[rgb]{0.64,0.00,0.00}{\textbf{#1}}}
\newcommand{\ExtensionTok}[1]{#1}
\newcommand{\FloatTok}[1]{\textcolor[rgb]{0.00,0.00,0.81}{#1}}
\newcommand{\FunctionTok}[1]{\textcolor[rgb]{0.00,0.00,0.00}{#1}}
\newcommand{\ImportTok}[1]{#1}
\newcommand{\InformationTok}[1]{\textcolor[rgb]{0.56,0.35,0.01}{\textbf{\textit{#1}}}}
\newcommand{\KeywordTok}[1]{\textcolor[rgb]{0.13,0.29,0.53}{\textbf{#1}}}
\newcommand{\NormalTok}[1]{#1}
\newcommand{\OperatorTok}[1]{\textcolor[rgb]{0.81,0.36,0.00}{\textbf{#1}}}
\newcommand{\OtherTok}[1]{\textcolor[rgb]{0.56,0.35,0.01}{#1}}
\newcommand{\PreprocessorTok}[1]{\textcolor[rgb]{0.56,0.35,0.01}{\textit{#1}}}
\newcommand{\RegionMarkerTok}[1]{#1}
\newcommand{\SpecialCharTok}[1]{\textcolor[rgb]{0.00,0.00,0.00}{#1}}
\newcommand{\SpecialStringTok}[1]{\textcolor[rgb]{0.31,0.60,0.02}{#1}}
\newcommand{\StringTok}[1]{\textcolor[rgb]{0.31,0.60,0.02}{#1}}
\newcommand{\VariableTok}[1]{\textcolor[rgb]{0.00,0.00,0.00}{#1}}
\newcommand{\VerbatimStringTok}[1]{\textcolor[rgb]{0.31,0.60,0.02}{#1}}
\newcommand{\WarningTok}[1]{\textcolor[rgb]{0.56,0.35,0.01}{\textbf{\textit{#1}}}}
\usepackage{longtable,booktabs}
\usepackage{graphicx,grffile}
\makeatletter
\def\maxwidth{\ifdim\Gin@nat@width>\linewidth\linewidth\else\Gin@nat@width\fi}
\def\maxheight{\ifdim\Gin@nat@height>\textheight\textheight\else\Gin@nat@height\fi}
\makeatother
% Scale images if necessary, so that they will not overflow the page
% margins by default, and it is still possible to overwrite the defaults
% using explicit options in \includegraphics[width, height, ...]{}
\setkeys{Gin}{width=\maxwidth,height=\maxheight,keepaspectratio}
\IfFileExists{parskip.sty}{%
\usepackage{parskip}
}{% else
\setlength{\parindent}{0pt}
\setlength{\parskip}{6pt plus 2pt minus 1pt}
}
\setlength{\emergencystretch}{3em}  % prevent overfull lines
\providecommand{\tightlist}{%
  \setlength{\itemsep}{0pt}\setlength{\parskip}{0pt}}
\setcounter{secnumdepth}{0}
% Redefines (sub)paragraphs to behave more like sections
\ifx\paragraph\undefined\else
\let\oldparagraph\paragraph
\renewcommand{\paragraph}[1]{\oldparagraph{#1}\mbox{}}
\fi
\ifx\subparagraph\undefined\else
\let\oldsubparagraph\subparagraph
\renewcommand{\subparagraph}[1]{\oldsubparagraph{#1}\mbox{}}
\fi

%%% Use protect on footnotes to avoid problems with footnotes in titles
\let\rmarkdownfootnote\footnote%
\def\footnote{\protect\rmarkdownfootnote}

%%% Change title format to be more compact
\usepackage{titling}

% Create subtitle command for use in maketitle
\providecommand{\subtitle}[1]{
  \posttitle{
    \begin{center}\large#1\end{center}
    }
}

\setlength{\droptitle}{-2em}

  \title{Rearranging Regional Data}
    \pretitle{\vspace{\droptitle}\centering\huge}
  \posttitle{\par}
    \author{Daniel Antal, CFA}
    \preauthor{\centering\large\emph}
  \postauthor{\par}
      \predate{\centering\large\emph}
  \postdate{\par}
    \date{9/19/2019}


\begin{document}
\maketitle

\hypertarget{working-with-eurostat-regional-data}{%
\subsection{Working with Eurostat Regional
Data}\label{working-with-eurostat-regional-data}}

The \href{http://ropengov.github.io/eurostat/}{eurostat} R package
creates an excellent framework to retrieve and analyze European
statistical data in a reproducible manner
\href{https://journal.r-project.org/archive/2017/RJ-2017-019/RJ-2017-019.pdf}{Leo
Lahti, Janne Huovari, Markus Kainu and Przemysław Biecek: Retrieval and
Analysis of Eurostat Open Data with the eurostat Package} It is not a
product that is anyway associated with Eurostat, but it provides a very
consistent approach to use Eurostat's statistical products.

\begin{figure}
\centering
\includegraphics{https://ec.europa.eu/eurostat/documents/345175/501899/Nuts-history}
\caption{NUTS}
\end{figure}

My supposedly reproducible research codes do not work from last year,
when the same products with the same IDs were reported under the
\texttt{NUTS2013} region boundary definitions. After spending countless
days with fixing my datasets, I learned a lot about the problematic
transition from \texttt{NUTS2013} to \texttt{NUTS2016} data products.
I'd like to create a package vignette, and possibly some helper
functions to save many hours of work for users of regional datasets.

\hypertarget{the-correspondence-table}{%
\subsection{The Correspondence Table}\label{the-correspondence-table}}

The Correspondence table is supposed to explain how NUTS regions have
changed and give guidance on what to do with the data. This is a
multi-sheet Excel file that is not tidy, and it is not complete. At the
time of writing this article, the changes in Slovenia and Greece are not
in the table.

\begin{Shaded}
\begin{Highlighting}[]
\CommentTok{#Download the latest Eurostat correspondence file.}
\CommentTok{#Change the subdirectory as you wish, I download it to 'data-raw'.}
\ControlFlowTok{if}\NormalTok{(}\OperatorTok{!}\StringTok{ }\KeywordTok{dir.exists}\NormalTok{(}\StringTok{'data-raw'}\NormalTok{)) }\KeywordTok{dir.create}\NormalTok{(}\StringTok{'data-raw'}\NormalTok{)}
\ControlFlowTok{if}\NormalTok{(}\OperatorTok{!}\StringTok{ }\KeywordTok{file.exists}\NormalTok{(}\KeywordTok{file.path}\NormalTok{(}\StringTok{'data-raw'}\NormalTok{ ,}\StringTok{'NUTS2013-NUTS2016.xlsx'}\NormalTok{))) \{}
  \KeywordTok{download.file}\NormalTok{ ( }\DataTypeTok{url =} \StringTok{'https://ec.europa.eu/eurostat/documents/345175/629341/NUTS2013-NUTS2016.xlsx'}\NormalTok{, }\DataTypeTok{destfile =} \KeywordTok{file.path}\NormalTok{(}\StringTok{'data-raw'}\NormalTok{, }\StringTok{'NUTS2013-NUTS2016.xlsx'}\NormalTok{ ))}
\NormalTok{\}}

\CommentTok{#The Excel file downloaded from Eurostat may or may not work on your system. If you are unlucky, you can try my copy. }
\ControlFlowTok{if}\NormalTok{(}\OperatorTok{!}\StringTok{ }\KeywordTok{file.exists}\NormalTok{(}\KeywordTok{file.path}\NormalTok{(}\StringTok{'data-raw'}\NormalTok{ ,}\StringTok{'NUTS2013-NUTS2016.xlsx'}\NormalTok{))) \{}
  \KeywordTok{download.file}\NormalTok{ ( }\DataTypeTok{url =} \StringTok{'https://github.com/antaldaniel/eurostat_regional/raw/master/data-raw/NUTS2013-NUTS2016.xlsx'}\NormalTok{, }\DataTypeTok{destfile =} \KeywordTok{file.path}\NormalTok{(}\StringTok{'data-raw'}\NormalTok{, }\StringTok{'NUTS2013-NUTS2016.xlsx'}\NormalTok{ ))}
\NormalTok{\}}

\NormalTok{regions <-}\StringTok{ }\NormalTok{readxl}\OperatorTok{::}\KeywordTok{read_excel}\NormalTok{( }\KeywordTok{file.path}\NormalTok{(}\StringTok{'data-raw'}\NormalTok{, }\StringTok{'NUTS2013-NUTS2016.xlsx'}\NormalTok{),}
                   \DataTypeTok{sheet =} \StringTok{'NUTS2013-NUTS2016'}\NormalTok{, }
                   \DataTypeTok{skip =} \DecValTok{1}\NormalTok{ , }\DataTypeTok{col_names =}\NormalTok{ T) }\OperatorTok
\StringTok{  }\KeywordTok{select}\NormalTok{ (}\DecValTok{1}\OperatorTok{:}\DecValTok{12}\NormalTok{) }\OperatorTok\StringTok{  }\CommentTok{#Extra unusued columns that will give a warning}
\StringTok{  }\NormalTok{purrr}\OperatorTok{::}\KeywordTok{set_names}\NormalTok{(., }\KeywordTok{c}\NormalTok{(}\StringTok{"rowid"}\NormalTok{, }\StringTok{"code13"}\NormalTok{, }\StringTok{"code16"}\NormalTok{, }
                        \StringTok{"country_name"}\NormalTok{, }\StringTok{'nuts1_name'}\NormalTok{, }\StringTok{'nuts2_name'}\NormalTok{,}
                        \StringTok{'nuts3_name'}\NormalTok{, }\StringTok{'change'}\NormalTok{, }\StringTok{'nuts_level'}\NormalTok{, }
                        \StringTok{'sort_countries'}\NormalTok{, }\StringTok{'sort_13'}\NormalTok{, }\StringTok{'sort_16'}\NormalTok{)) }\OperatorTok
\StringTok{  }\KeywordTok{mutate}\NormalTok{ ( }\DataTypeTok{name =} \KeywordTok{case_when}\NormalTok{ ( }
    \OperatorTok{!}\KeywordTok{is.na}\NormalTok{(nuts1_name) }\OperatorTok{~}\StringTok{ }\NormalTok{nuts1_name, }
    \OperatorTok{!}\KeywordTok{is.na}\NormalTok{(nuts2_name) }\OperatorTok{~}\StringTok{ }\NormalTok{nuts2_name,}
    \OperatorTok{!}\KeywordTok{is.na}\NormalTok{(nuts3_name) }\OperatorTok{~}\StringTok{ }\NormalTok{nuts3_name, }
    \OperatorTok{!}\KeywordTok{is.na}\NormalTok{(country_name)}\OperatorTok{~}\StringTok{ }\NormalTok{country_name,}
    \OtherTok{TRUE} \OperatorTok{~}\StringTok{ }\OtherTok{NA_character_}\NormalTok{))}
\end{Highlighting}
\end{Shaded}

\begin{verbatim}
## New names:
## * `` -> ...1
## * `` -> ...13
## * `` -> ...14
## * `` -> ...15
\end{verbatim}

\begin{Shaded}
\begin{Highlighting}[]
\CommentTok{#Our reference vector are the NUTS2016 codes}
\NormalTok{nuts_}\DecValTok{2016}\NormalTok{_codes <-}\StringTok{ }\KeywordTok{unique}\NormalTok{ (regions}\OperatorTok{$}\NormalTok{code16)}
\end{Highlighting}
\end{Shaded}

It is a good idea to have a comprehensive look of what happened, because
each data set may have a different problem. I organized the regions into
\texttt{unchanged\_regions}, \texttt{changed\_regions} and
\texttt{discontinued\_regions}. These tables will help in correcting the
data. Unfortunately, I found some regions in Slovenia and Greece that
are not in any of these categories.

\begin{Shaded}
\begin{Highlighting}[]
\CommentTok{##In these cases, the code13 == code16}
\NormalTok{unchanged_regions <-}\StringTok{ }\NormalTok{regions }\OperatorTok\StringTok{ }\KeywordTok{filter}\NormalTok{ ( }\KeywordTok{is.na}\NormalTok{(change)) }\OperatorTok
\StringTok{  }\KeywordTok{fill}\NormalTok{ ( nuts1_name ) }\OperatorTok
\StringTok{  }\KeywordTok{fill}\NormalTok{ ( nuts2_name ) }\OperatorTok
\StringTok{  }\KeywordTok{select}\NormalTok{ ( code13, code16, name )}

\CommentTok{## In these cases code13 != code16}
\NormalTok{changed_regions <-}\StringTok{ }\NormalTok{regions }\OperatorTok\StringTok{ }\KeywordTok{filter}\NormalTok{ ( }\OperatorTok{!}\KeywordTok{is.na}\NormalTok{(change)) }\OperatorTok
\StringTok{  }\KeywordTok{fill}\NormalTok{ ( nuts1_name ) }\OperatorTok
\StringTok{  }\KeywordTok{fill}\NormalTok{ ( nuts2_name ) }\OperatorTok
\StringTok{  }\KeywordTok{select}\NormalTok{ ( code13, code16, name, nuts_level, change )}

\NormalTok{discontinued_regions  <-}\StringTok{ }\NormalTok{changed_regions }\OperatorTok
\StringTok{  }\KeywordTok{filter}\NormalTok{ ( change }\OperatorTok{==}\StringTok{ "discontinued"}\NormalTok{)}
\end{Highlighting}
\end{Shaded}

\hypertarget{nuts1-correspondence}{%
\subsubsection{NUTS1 Correspondence}\label{nuts1-correspondence}}

\begin{Shaded}
\begin{Highlighting}[]
\NormalTok{nuts1_correspondence <-}\StringTok{ }\NormalTok{readxl}\OperatorTok{::}\KeywordTok{read_excel}\NormalTok{( }\KeywordTok{file.path}\NormalTok{(}\StringTok{'data-raw'}\NormalTok{, }\StringTok{'NUTS2013-NUTS2016.xlsx'}\NormalTok{),}
 \DataTypeTok{sheet =} \StringTok{'Correspondence NUTS-1'}\NormalTok{, }
 \DataTypeTok{skip =} \DecValTok{0}\NormalTok{ , }\DataTypeTok{col_names =}\NormalTok{ T) }\OperatorTok
\StringTok{  }\KeywordTok{set_names}\NormalTok{ ( ., }\KeywordTok{c}\NormalTok{(}\StringTok{"code13"}\NormalTok{, }\StringTok{"code16"}\NormalTok{, }\StringTok{"nuts1_name"}\NormalTok{, }\StringTok{"change"}\NormalTok{, }\StringTok{"resolution"}\NormalTok{)) }\OperatorTok
\StringTok{  }\KeywordTok{mutate_if}\NormalTok{ ( is.factor, as.character )}

\NormalTok{knitr}\OperatorTok{::}\KeywordTok{kable}\NormalTok{ (}\KeywordTok{head}\NormalTok{(nuts1_correspondence,}\DecValTok{6}\NormalTok{))}
\end{Highlighting}
\end{Shaded}

\begin{longtable}[]{@{}lllll@{}}
\toprule
code13 & code16 & nuts1\_name & change & resolution\tabularnewline
\midrule
\endhead
NA & FRB & CENTRE --- VAL DE LOIRE & new NUTS 1 region, identical to
ex-NUTS 2 region FR24 & FRB=FR24\tabularnewline
NA & FRC & BOURGOGNE-FRANCHE-COMTÉ & new NUTS 1 region, merge of ex-NUTS
2 regions FR26 and FR43 & FRC=FR26+FR43\tabularnewline
NA & FRD & NORMANDIE & new NUTS 1 region, merge of ex-NUTS 2 regions
FR23 and FR25 & FRD=FR23+FR25\tabularnewline
NA & FRE & NORD-PAS DE CALAIS-PICARDIE & new NUTS 1 region, merge of
ex-NUTS 2 regions FR22 and FR30 & FRE=FR22+FR30\tabularnewline
NA & FRF & ALSACE-CHAMPAGNE-ARDENNE-LORRAINE & new NUTS 1 region, merge
of ex-NUTS 2 regions FR 21, FR 41 and FR42 &
FRF=FR21+FR41+FR42\tabularnewline
NA & FRG & PAYS DE LA LOIRE & new NUTS 1 region, identical to ex-NUTS 2
region FR51 & FRG=FR51\tabularnewline
\bottomrule
\end{longtable}

\hypertarget{nuts2-correspondence}{%
\subsubsection{NUTS2 Correspondence}\label{nuts2-correspondence}}

Similarly, there is a table for NUTS2 level. You have to beware that all
NUTS2 regions are contained in wider NUTS1 regions. So there are two
logically possible cases: a NUTS2 change affected only one NUTS1 region
(changes were made within a larger region), or it affected several NUTS1
regions. If you work with data that has both NUTS1 and NUTS2 level
information, such as any of the internet useage products, for example,
\texttt{isoc\_r\_iuse\_i}, you must correct NUTS1 and NUTS2 level
problems consistently.

\begin{Shaded}
\begin{Highlighting}[]
\NormalTok{nuts2_correspondence <-}\StringTok{ }\NormalTok{readxl}\OperatorTok{::}\KeywordTok{read_excel}\NormalTok{( }
  \KeywordTok{file.path}\NormalTok{(}\StringTok{'data-raw'}\NormalTok{, }\StringTok{'NUTS2013-NUTS2016.xlsx'}\NormalTok{),                                            }\DataTypeTok{sheet =} \StringTok{'Correspondence NUTS-2'}\NormalTok{,                                            }\DataTypeTok{skip =} \DecValTok{0}\NormalTok{ , }\DataTypeTok{col_names =} \OtherTok{TRUE}\NormalTok{) }\OperatorTok
\StringTok{  }\KeywordTok{select}\NormalTok{ ( }\DecValTok{1}\OperatorTok{:}\DecValTok{5}\NormalTok{ ) }\OperatorTok
\StringTok{  }\KeywordTok{set_names}\NormalTok{ ( ., }\KeywordTok{c}\NormalTok{(}\StringTok{"code13"}\NormalTok{, }\StringTok{"code16"}\NormalTok{, }\StringTok{"nuts1_name"}\NormalTok{,}
                   \StringTok{"change"}\NormalTok{, }\StringTok{"resolution"}\NormalTok{)) }\OperatorTok
\StringTok{  }\KeywordTok{filter}\NormalTok{ ( }\KeywordTok{is.na}\NormalTok{(code13) }\OperatorTok{+}\StringTok{ }\KeywordTok{is.na}\NormalTok{(code16) }\OperatorTok{<}\StringTok{ }\DecValTok{2}\NormalTok{) }\CommentTok{#The table has empty rows}
\end{Highlighting}
\end{Shaded}

\begin{verbatim}
## New names:
## * `` -> ...6
## * `` -> ...7
## * `` -> ...8
\end{verbatim}

\begin{Shaded}
\begin{Highlighting}[]
\NormalTok{knitr}\OperatorTok{::}\KeywordTok{kable}\NormalTok{ (}\KeywordTok{head}\NormalTok{(nuts2_correspondence,}\DecValTok{6}\NormalTok{))}
\end{Highlighting}
\end{Shaded}

\begin{longtable}[]{@{}lllll@{}}
\toprule
code13 & code16 & nuts1\_name & change & resolution\tabularnewline
\midrule
\endhead
IE01 & NA & Border, Midland and Western & discontinued &
NA\tabularnewline
NA & IE04 & Northern and Western & new region & NA\tabularnewline
IE02 & NA & Southern and Eastern & discontinued & NA\tabularnewline
NA & IE05 & Southern & new region, made from ex-IE023, IE024 and IE025 &
IE05=IE023+IE024+IE025\tabularnewline
NA & IE06 & Eastern and Midland & new region & NA\tabularnewline
FR24 & FRB0 & Centre --- Val de Loire & recoded and relabelled &
FRB0=FR24\tabularnewline
\bottomrule
\end{longtable}

\hypertarget{nuts3-correspondence}{%
\subsubsection{NUTS3 Correspondence}\label{nuts3-correspondence}}

The NUTS3 regions are small regions within the NUTS2 regions. There are
only a few NUTS3 level statistical products available, and in the
absence of an even lower resolution, most changes at this level cannot
be reconciled.

However, many corrections on NUTS1 or NUTS2 level require reconciliation
with the use of NUTS3 data, which, sadly, contains a lot of relabelling
and other changes. So if you really want to fix everything, you have to
dwell into the NUTS3 level, too.

\begin{Shaded}
\begin{Highlighting}[]
\NormalTok{nuts3_correspondence <-}\StringTok{ }\NormalTok{readxl}\OperatorTok{::}\KeywordTok{read_excel}\NormalTok{( }
  \KeywordTok{file.path}\NormalTok{(}\StringTok{'data-raw'}\NormalTok{, }\StringTok{'NUTS2013-NUTS2016.xlsx'}\NormalTok{),                                            }\DataTypeTok{sheet =} \StringTok{'Correspondence NUTS-3'}\NormalTok{,                                            }\DataTypeTok{skip =} \DecValTok{0}\NormalTok{ , }\DataTypeTok{col_names =} \OtherTok{TRUE}\NormalTok{) }\OperatorTok
\StringTok{  }\KeywordTok{select}\NormalTok{ ( }\DecValTok{1}\OperatorTok{:}\DecValTok{5}\NormalTok{ ) }\OperatorTok
\StringTok{  }\KeywordTok{set_names}\NormalTok{ ( ., }\KeywordTok{c}\NormalTok{(}\StringTok{"code13"}\NormalTok{, }\StringTok{"code16"}\NormalTok{, }\StringTok{"nuts1_name"}\NormalTok{,}
                   \StringTok{"change"}\NormalTok{, }\StringTok{"resolution"}\NormalTok{)) }\OperatorTok
\StringTok{  }\KeywordTok{filter}\NormalTok{ ( }\KeywordTok{is.na}\NormalTok{(code13) }\OperatorTok{+}\StringTok{ }\KeywordTok{is.na}\NormalTok{(code16) }\OperatorTok{<}\StringTok{ }\DecValTok{2}\NormalTok{) }\CommentTok{#The table has empty rows}

\NormalTok{knitr}\OperatorTok{::}\KeywordTok{kable}\NormalTok{ (}\KeywordTok{head}\NormalTok{(nuts3_correspondence,}\DecValTok{6}\NormalTok{))}
\end{Highlighting}
\end{Shaded}

\begin{longtable}[]{@{}lllll@{}}
\toprule
code13 & code16 & nuts1\_name & change & resolution\tabularnewline
\midrule
\endhead
DE915 & NA & Göttingen & discontinued; merged with ex-DE919 &
NA\tabularnewline
DE919 & NA & Osterode am Harz & discontinued; merged with ex-DE915 &
NA\tabularnewline
NA & DE91C & Göttingen & new region, merge of ex-DE915 and DE919 &
DE91C=DE915+DE919\tabularnewline
DEB16 & DEB1C & Cochem-Zell & boundary shift & NA\tabularnewline
DEB19 & DEB1D & Rhein-Hunsrück-Kreis & boundary shift &
NA\tabularnewline
IE011 & IE041 & Border & boundary shift & NA\tabularnewline
\bottomrule
\end{longtable}

\hypertarget{problems-from-inconsistent-nuts-level-data}{%
\subsection{Problems from inconsistent NUTS level
data}\label{problems-from-inconsistent-nuts-level-data}}

\hypertarget{inconsistent-levels}{%
\subsubsection{Inconsistent levels}\label{inconsistent-levels}}

In many cases, the name and/or the metadata description of the
statistical product refers to NUTS2 level data, but in fact, you receive
a mixed data table that has data from all NUTS levels.

This may create some extra filtering work, especially if you want to be
reproducible, tidy, or just want to check consistencies, but this can be
a fortunate arrangement that Eurostat should keep and expand in the
future with better metadata. In this case, usually you have a lot of
information present to correct the some problems.

For example, in
\href{https://appsso.eurostat.ec.europa.eu/nui/show.do?dataset=isoc_r_iuse_i\&lang=en}{isoc\_r\_iuse\_i},
you find the following geo units that do not conform the current,
NUTS2016 definion, and you cannot correct them on the basis of the
correspondence table, because it does not contain in the current version
information about Slovenia and Greece: SI01,SI02,EL1,EL2.

However, the following regions can be corrected: FRB, FRC, FRD, FRE,
FRF, FRG, FRH, FRI, FRJ, FRK, FRL, FRM, FRY, PL2, PL4, PL5, PL6, PL7,
PL8, PL9, IE04, IE05, IE06, FRB0, FRC1, FRC2, FRD1, FRD2, FRE1, FRE2,
FRF1, FRF2, FRF3, FRG0, FRH0, FRI1, FRI2, FRI3, FRJ1, FRJ2, FRK1, FRK2,
FRL0, FRM0, FRY1, FRY2, FRY3, FRY4, LT01, LT02, HU11, HU12.

In this case, with care you can add 4 NUTS1 regions to your dataset.

The usual problem with information society indicators is that they do
not contain NUTS2 level information for larger countries. In order to
avoid unnecessary data loss, Eurostat (correctly) decided to report
NUTS1 level data for large countries and NUTS2 level data for smaller
countries. As far as I know the reason is that the microdata are surveys
with fixed 1000-1500 national sample size, which would allow a regional
breakup in the case of Estonia on NUTS3 level but only at NUTS1 level in
Great Britain and Germany. The way the samples are designed that they
must be representative for Northern Ireland, which is a NUTS2 region,
and therefore in the case of the UK you have inconsistent NUTS levels
within a single country.

While the best representation of the data is what Eurostat publishes, a
lot of warnings would be required here in the title of the product and
the metadata. If you start to join this nominally NUTS2 datasets with
other NUTS2 data, you will immediately loose the large member states!
This is actually a very strong case to make all regional data products
cross-level, i.e.~whenever a data is avaiable on NUTS2 level, it should
be reported in the same product on NUTS1 and NUTS0 levels, too. {[}\^{}I
will come back to this problem later.{]}

\hypertarget{small-country-problems}{%
\subsubsection{Small country problems}\label{small-country-problems}}

Malta and Luxembourg are much smaller than the provinces of Germany, or
the British home countries Northern Ireland or Wales. Statistical
regions were designed to create more homogeneous territorial units in
size and other characteristics. This means that countries smaller than a
usual NUTS1 region in larger countries do not have a NUTS1 level
division. Very small countries do not have a NUTS2 division. Cyprus,
Estonia, Luxembourg, Malta have only NUTS3 divisions.

So, the rule of NUTS regions is NUTS0 \textgreater{}= NUTS1
\textgreater{}= NUTS2 \textgreater{} NUTS3, where the NUTS0 level stands
for the largest territorial division, which is national boundary.

In this case, \texttt{LU} = \texttt{LU0} = \texttt{LU00}. If you want to
have a complete map of NUTS2 regions of Europe, obviously you have to
treat Luxembourg and the other small member states as NUTS2 regions.
Provided that you have the data!

There is an inconsistency how the data is present in the datasets, but
you can almost always find the \texttt{LU} code. If \texttt{LU0} is
missing, you can just create a copy of the \texttt{LU} data and label it
\texttt{LU0} on NUTS1 level, and repeat the same as \texttt{LU00} on
NUTS2 level. It also follows that NUTS0 codes are always made of 2
characters, NUTS1 is always two alphabetic characters followed by a
number, and NUTS2 is always two alphabetic followed by 2 numbers.

So far I have shown problems that can be solved with simple filtering or
relabelling if you understand the hiearchy of NUTS levels. A far more
serious problem when NUTS2013 and NUTS2016 data got mixed without proper
metadata.

\hypertarget{problems-of-mislabelling}{%
\subsection{Problems of Mislabelling}\label{problems-of-mislabelling}}

A typical mislabelling problem is that in the same data/metadata column
\texttt{geo} you find mixed region codes, following seemingly randomly
\texttt{NUTS2013} and \texttt{NUTS2016} codes. This is a real error, and
it requires more than data tiding to fix these problems, up to the level
they can be fixed.

\hypertarget{recoded-regions}{%
\subsubsection{Recoded regions}\label{recoded-regions}}

The simplest error is a recoding error. You will most likely run into
this problem with France, a country that comprehensively changes its
NUTS units. To avoid confusion, the French altered the alphanumeric
codes completely, and replaced the letter:letter:number structure of the
NUTS codes to letter:letter:letter, like instead of using an
\texttt{FR7} like nomenclature, they use a hexadecimal-like
\texttt{FRK}. This is a wise move but it is not followed up by the
Eurostat data.

In many cases, the recoding means that the region did not change, but
the abbreviated code did. For example, the change is merely
\texttt{FRK1=FR72}. You have to look up all instances of \texttt{FR72}
labels in your data, and if they are present, change them to
\texttt{FRK1}.

In some cases, the name of the regionn changed (should you use them in
your table, beware), but the region itself is the same. For example, the
name of \texttt{MAKROREGION\ PÓŁNOCNO-ZACHODNI} was changed, but its
code, \texttt{PL4} and its boundaries remained the same. Because of many
possible character encoding problems, I do not recommend the use of
names in statistical analysis. If you need them in visualizations, add
them back in the last, visualiztion stage.

For example, the
\href{https://ec.europa.eu/eurostat/web/products-datasets/product?code=tgs00096}{tgs00096}
has consistently wrong labels for all French NUTS2 regions. The data in
this product (Population on 1 January by NUTS 2 region) goes back to
pre-NUTS2013 definitions, so this is not purely the same problem. The
earliest data in this product is from 2007, so four NUTS definition
changes may affect them. (See the
\href{https://ec.europa.eu/eurostat/web/nuts/history}{history of NUTS}
on the Eurostat website.)

\hypertarget{discountinued-regions}{%
\subsubsection{Discountinued regions}\label{discountinued-regions}}

Many regions are discontinued, so you may have data from the
\texttt{NUTS2013} definition that you will not use in a
\texttt{NUTS2016} map or analysis. Do not discard these data points yet,
because you may be able to use them to re-create some missing
\texttt{NUTS2016} observations.

\hypertarget{changed-regions}{%
\subsubsection{Changed regions}\label{changed-regions}}

In the correspondence files, you find hints how to solve missing case
problems. For example, if \texttt{IE05} is missing from your NUTS2
table, you can try to impute it with the equation
\texttt{IE05=IE023+IE024+IE025}. In this particular case, because
\texttt{IE023} is a NUTS3 small regions's code, you have to go down to
that particular level to impute the missing NUTS2 case. If you are
lucky, the NUTS3 data is in the dataset (incorrectly labelled as NUTS2),
or it is available from a NUTS3-level product.

There are a few extra caveats here:

\begin{itemize}
\item
  You can technically do the imputation if the change does not mean a
  change of NUTS3 boundaries, only a rearrangement of constituent lower
  level boundaries. In this case, \emph{the changes are additive}, and
  you can impute the data \emph{provided the data itself is additive}.
  You can add together population numbers or euro figures.
\item
  You cannot get the missing data with addition and subtraction in the
  case the data is not atomic, and therefore it is not additive. You
  cannot add or subtracted population density or euro per capita
  figures. In this case, you can only do the imputation if you
  additively come up with euro and population figures for
  \texttt{IE023+IE024+IE025}, and divide the result.
\item
  If the change is not a re-arrangement of lower level, lower resolution
  data, than you cannot come to any precise formula to fill in the
  missing observation.
\item
  You will not be able to fill the missing observation when any of the
  constituent data is unavailable. You will not be able to get the
  \texttt{IE05} data if any of \texttt{IE023} or \texttt{IE024} or
  \texttt{IE025} is missing.
\end{itemize}

\hypertarget{special-territorial-units}{%
\subsection{Special territorial units}\label{special-territorial-units}}

\hypertarget{extraterritorial-units}{%
\subsubsection{Extraterritorial units}\label{extraterritorial-units}}

Some NUTS codes end with \texttt{Z}, such as \texttt{ITZ}, \texttt{ITZZ}
and \texttt{ITZZZ}, in this case, extra-regional data for Italy. You can
treat this as a kind of a correction raw, and it usually, or probably
always empty in the case of regional statistics. It is used for data
that cannot be connected to a region.

However, the fact that they are always \texttt{NA} values can cause
problems with imputation algorithms, so the best practice is to filter
these empty rows out.

\hypertarget{non-eu-member-states}{%
\subsubsection{Non-EU member states}\label{non-eu-member-states}}

It is a welcome development that more and more regional statistics
include EEA members, candidate or potential candidate countries.
Sometimes you can find regional data about Albania, Andora,
Bosnia-Herzegovina, Iceland, Kosovo, Lichtenstein, Montenegro, North
Macedonia, Serbia, Switzerland, Turkey and Norway, and probably further
states, too.

If the coding is correct, you can easily put the data on a map, too,
because the maps that are used by the \texttt{eurostat} package contain
the polygons of these regions. However, the NUTS correspondence table do
not include them, so there is no recourse to check their consistency.
They sometimes pop up, sometimes not, so make sure if you have too many
observations, check for them.

\hypertarget{one-possible-solution}{%
\subsection{One possible solution}\label{one-possible-solution}}

The following R code is based on the
\href{https://www.tidyverse.org/}{tidyverse} packages and uses
non-standard evaluation. This is a possible solution to solve (most of)
the problem.

You need to run the first chunks to make the Eurostat correspondence
table tidy (See:
\href{https://vita.had.co.nz/papers/tidy-data.pdf}{Hadley Wickham: Tidy
Data}).

As you can see, this is not for the faint heart, and I do not guarrantee
that it works with all data perfectly. It is more of an illustration of
the problem and a list of ideas how you can improve the data quality if
you have to work with historical data or with panel data. I do not think
that this should be included in the eurostat package, rather a
constructive discussion should start with Eurostat.

The following function requires to bring the Eurostat regional data into
a tidy form. The data that you get with
\texttt{eurostat::get\_eurostat(id\ =\ \textquotesingle{}demo\_r\_d3area\textquotesingle{})}
is a quasi-long-form data. It is not fully tidy wide, and not fully tidy
long.

What I do is that I keep \texttt{geo}, \texttt{time}, \texttt{values},
add a \texttt{description}, create and ISO-conform
\texttt{country\_code} and \texttt{country} column, and in the case of
annual data for ease of use, I take the year part out of the
\texttt{time} variable to \texttt{years}.

Furthermore, whatever variables are left, for example, \texttt{unit},
\texttt{landuse}, \texttt{cofog}, \texttt{indic\_is},
\texttt{ind\_type}, etc, are united into a single, unique column name,
which I call \texttt{indicator}. So, if you want to run my code without
modifications, do that, but of course, you can use your own column
structures.

The idea of course is to end up with a table where each unique
\texttt{geo} and \texttt{time} and\texttt{indicator} have one and only
one \texttt{value}. This is absolutely necessary to clean up the data,
and later to join the data with maps or other data.

\begin{Shaded}
\begin{Highlighting}[]
\KeywordTok{require}\NormalTok{(tidyverse)}
\CommentTok{#You need the metadata information from the first chunk if you run this,}
\CommentTok{#i.e. you must read the Excel correspondence tables and tidy them up.}

\NormalTok{correct_nuts_labelling <-}\StringTok{ }\ControlFlowTok{function}\NormalTok{ ( dat ) \{ }

\NormalTok{  tmp <-}\StringTok{ }\NormalTok{dat  }\OperatorTok
\StringTok{    }\KeywordTok{mutate_if}\NormalTok{ ( is.factor, as.character ) }\OperatorTok
\StringTok{    }\KeywordTok{left_join}\NormalTok{ ( unchanged_regions }\OperatorTok\StringTok{ }
\StringTok{                  }\KeywordTok{select}\NormalTok{ ( code16 ) }\OperatorTok
\StringTok{                  }\KeywordTok{rename}\NormalTok{ ( }\DataTypeTok{geo =}\NormalTok{ code16 ) }\OperatorTok
\StringTok{                  }\KeywordTok{mutate}\NormalTok{ ( }\DataTypeTok{change =} \StringTok{'unchanged'}\NormalTok{), }\DataTypeTok{by =} \StringTok{'geo'}\NormalTok{) }\OperatorTok
\StringTok{    }\KeywordTok{mutate}\NormalTok{ ( }\DataTypeTok{change  =} \KeywordTok{ifelse}\NormalTok{ (  country_code }\OperatorTok\StringTok{ }\KeywordTok{c}\NormalTok{(}\StringTok{"CH"}\NormalTok{, }\StringTok{"TR"}\NormalTok{, }\StringTok{"NO"}\NormalTok{,}
                                                     \StringTok{"MK"}\NormalTok{, }\StringTok{"IS"}\NormalTok{, }\StringTok{"LI"}\NormalTok{, }
                                                     \StringTok{"AD"}\NormalTok{, }\StringTok{"SM"}\NormalTok{, }\StringTok{"VA"}\NormalTok{, }
                                                     \StringTok{"XK"}\NormalTok{, }\StringTok{"RS"}\NormalTok{, }\StringTok{"ME"}\NormalTok{,}
                                                     \StringTok{"BA"}\NormalTok{, }\StringTok{"AL"}\NormalTok{), }
                                 \StringTok{'not_EU'}\NormalTok{, change)) }\OperatorTok
\StringTok{    }\KeywordTok{mutate_if}\NormalTok{ ( is.factor, as.character ) }\OperatorTok
\StringTok{    }\KeywordTok{filter}\NormalTok{ ( stringr}\OperatorTok{::}\KeywordTok{str_sub}\NormalTok{(geo, }\DecValTok{-3}\NormalTok{,}\OperatorTok{-}\DecValTok{1}\NormalTok{) }\OperatorTok{!=}\StringTok{ "ZZZ"}\NormalTok{, }
\NormalTok{             stringr}\OperatorTok{::}\KeywordTok{str_sub}\NormalTok{(geo, }\DecValTok{-2}\NormalTok{,}\OperatorTok{-}\DecValTok{1}\NormalTok{) }\OperatorTok{!=}\StringTok{ "ZZ"}\NormalTok{)}
  
\NormalTok{  tmp_eu_only <-}\StringTok{ }\NormalTok{tmp }\OperatorTok\StringTok{ }\KeywordTok{filter}\NormalTok{ ( change }\OperatorTok{!=}\StringTok{ "not_eu"}\NormalTok{)}
  
  \CommentTok{##The ZZZ parts filter out data that is not related to NUTS territorial}
  \CommentTok{##units and likely to be missing. This can avoid a lot of compliation}
  \CommentTok{##during imputation or mapping.}
  
\NormalTok{  missing_}\DecValTok{2016}\NormalTok{_codes <-}\StringTok{ }\NormalTok{nuts_}\DecValTok{2016}\NormalTok{_codes [}\KeywordTok{which}\NormalTok{ (}\OperatorTok{!}\StringTok{ }\NormalTok{nuts_}\DecValTok{2016}\NormalTok{_codes }\OperatorTok\StringTok{ }\NormalTok{tmp_eu_only}\OperatorTok{$}\NormalTok{geo )]}
\NormalTok{  missing_}\DecValTok{2016}\NormalTok{_codes <-}\StringTok{ }\NormalTok{missing_}\DecValTok{2016}\NormalTok{_codes [ }\KeywordTok{which}\NormalTok{ (stringr}\OperatorTok{::}\KeywordTok{str_sub}\NormalTok{(missing_}\DecValTok{2016}\NormalTok{_codes, }\DecValTok{-3}\NormalTok{, }\DecValTok{-1}\NormalTok{) }\OperatorTok{!=}\StringTok{ "ZZZ"}\NormalTok{)]}
\NormalTok{  missing_}\DecValTok{2016}\NormalTok{_codes <-}\StringTok{ }\NormalTok{missing_}\DecValTok{2016}\NormalTok{_codes [ }\KeywordTok{which}\NormalTok{ (stringr}\OperatorTok{::}\KeywordTok{str_sub}\NormalTok{(missing_}\DecValTok{2016}\NormalTok{_codes, }\DecValTok{-2}\NormalTok{, }\DecValTok{-1}\NormalTok{) }\OperatorTok{!=}\StringTok{ "ZZ"}\NormalTok{)]}
  
  \CommentTok{#Here are you missing NUTS1 and NUTS2 units.  }
  \CommentTok{#If you want to, you further add NUTS3 here to the code.}
\NormalTok{  missing_nuts1_}\DecValTok{2016}\NormalTok{ <-}\StringTok{ }\NormalTok{missing_}\DecValTok{2016}\NormalTok{_codes [ }\KeywordTok{which}\NormalTok{ (}\KeywordTok{nchar}\NormalTok{(missing_}\DecValTok{2016}\NormalTok{_codes) }\OperatorTok{==}\StringTok{ }\DecValTok{3}\NormalTok{)]}
\NormalTok{  missing_nuts2_}\DecValTok{2016}\NormalTok{ <-}\StringTok{ }\NormalTok{missing_}\DecValTok{2016}\NormalTok{_codes [ }\KeywordTok{which}\NormalTok{ (}\KeywordTok{nchar}\NormalTok{(missing_}\DecValTok{2016}\NormalTok{_codes) }\OperatorTok{==}\StringTok{ }\DecValTok{4}\NormalTok{)]}
  
  
\NormalTok{  tmp2 <-}\StringTok{ }\NormalTok{tmp }\OperatorTok
\StringTok{    }\KeywordTok{left_join}\NormalTok{ ( changed_regions }\OperatorTok\StringTok{ }
\StringTok{                  }\KeywordTok{select}\NormalTok{ ( code16, change ) }\OperatorTok
\StringTok{                  }\KeywordTok{rename}\NormalTok{ ( }\DataTypeTok{geo =}\NormalTok{ code16 ), }
                \DataTypeTok{by =} \KeywordTok{c}\NormalTok{(}\StringTok{"geo"}\NormalTok{, }\StringTok{"change"}\NormalTok{)) }
  
  \CommentTok{## This is the data that can be used without risk}
\NormalTok{  correctly_labelled_unchanged <-}\StringTok{ }\NormalTok{tmp2 }\OperatorTok\StringTok{ }\KeywordTok{filter}\NormalTok{ ( }\OperatorTok{!}\KeywordTok{is.na}\NormalTok{(change))}
  
\NormalTok{  tmp3 <-}\StringTok{ }\NormalTok{tmp2 }\OperatorTok\StringTok{ }\KeywordTok{filter}\NormalTok{ ( }\KeywordTok{is.na}\NormalTok{( change )) }
  
  \CommentTok{## These are the regions that were changed but they are}
  \CommentTok{## correctly labelled}
\NormalTok{  correctly_labelled_changed <-}\StringTok{ }\NormalTok{tmp3 }\OperatorTok\StringTok{ }
\StringTok{    }\KeywordTok{filter}\NormalTok{ ( geo }\OperatorTok\StringTok{ }\NormalTok{changed_regions}\OperatorTok{$}\NormalTok{code16 )}
  
  \CommentTok{## There are the incorrectly labelled observations in your }
  \CommentTok{## dataset.}
\NormalTok{  incorrectlly_labelled_nuts13 <-}\StringTok{ }\NormalTok{tmp3 }\OperatorTok
\StringTok{    }\KeywordTok{filter}\NormalTok{ ( geo }\OperatorTok\StringTok{ }\NormalTok{changed_regions}\OperatorTok{$}\NormalTok{code13 )}
  
  \CommentTok{## You need to treat the NUTS1 level first.}
\NormalTok{  incorrectly_labelled_nuts1_}\DecValTok{2013}\NormalTok{ <-}\StringTok{ }\NormalTok{incorrectlly_labelled_nuts13 }\OperatorTok
\StringTok{    }\KeywordTok{filter}\NormalTok{ ( }\KeywordTok{nchar}\NormalTok{ (}\KeywordTok{as.character}\NormalTok{(geo)) }\OperatorTok{==}\StringTok{ }\DecValTok{3}\NormalTok{) }\OperatorTok
\StringTok{    }\KeywordTok{select}\NormalTok{ ( }\OperatorTok{-}\NormalTok{change ) }\OperatorTok
\StringTok{    }\KeywordTok{left_join}\NormalTok{ ( nuts1_correspondence }\OperatorTok
\StringTok{                  }\KeywordTok{rename}\NormalTok{ ( }\DataTypeTok{geo =}\NormalTok{ code13 ) }\OperatorTok
\StringTok{                  }\KeywordTok{filter}\NormalTok{ ( }\OperatorTok{!}\KeywordTok{is.na}\NormalTok{(geo)) }\OperatorTok
\StringTok{                  }\KeywordTok{select}\NormalTok{ ( geo, code16, change, resolution ), }
                \DataTypeTok{by =} \StringTok{'geo'}\NormalTok{) }\OperatorTok
\StringTok{    }\KeywordTok{filter}\NormalTok{ ( change }\OperatorTok{!=}\StringTok{ "discontinued"}\NormalTok{) }\OperatorTok
\StringTok{    }\KeywordTok{mutate}\NormalTok{ ( }\DataTypeTok{problem_code =}\NormalTok{ geo ) }\OperatorTok
\StringTok{    }\KeywordTok{mutate}\NormalTok{  ( }\DataTypeTok{geo =}\NormalTok{  code16)}
  
\NormalTok{  nuts1_missings <-}\StringTok{ }\NormalTok{missing_nuts1_}\DecValTok{2016}\NormalTok{  [ }\KeywordTok{which}\NormalTok{ ( missing_nuts1_}\DecValTok{2016} \OperatorTok\StringTok{ }\NormalTok{incorrectly_labelled_nuts1_}\DecValTok{2013}\OperatorTok{$}\NormalTok{geo)] }

  \CommentTok{## Probably you have recovered some NUTS1 regions.}
\NormalTok{  found_nuts1 <-}\StringTok{ }\NormalTok{incorrectly_labelled_nuts1_}\DecValTok{2013} \OperatorTok
\StringTok{    }\KeywordTok{filter}\NormalTok{ (  geo }\OperatorTok\StringTok{ }\NormalTok{missing_nuts1_}\DecValTok{2016}\NormalTok{  )}
  \KeywordTok{message}\NormalTok{ ( }\StringTok{"Found "}\NormalTok{, }\KeywordTok{length}\NormalTok{(}\KeywordTok{unique}\NormalTok{(found_nuts1}\OperatorTok{$}\NormalTok{geo)), }\StringTok{" NUTS1 regions"}\NormalTok{)}
  
  \CommentTok{## Repeat to NUTS2 regions.}
\NormalTok{  incorrectly_labelled_nuts2_}\DecValTok{2013}\NormalTok{ <-}\StringTok{ }\NormalTok{incorrectlly_labelled_nuts13 }\OperatorTok
\StringTok{    }\KeywordTok{filter}\NormalTok{ ( }\KeywordTok{nchar}\NormalTok{ (}\KeywordTok{as.character}\NormalTok{(geo)) }\OperatorTok{==}\StringTok{ }\DecValTok{4}\NormalTok{) }\OperatorTok
\StringTok{    }\KeywordTok{select}\NormalTok{ ( }\OperatorTok{-}\NormalTok{change ) }\OperatorTok
\StringTok{    }\KeywordTok{left_join}\NormalTok{ ( nuts2_correspondence }\OperatorTok
\StringTok{                  }\KeywordTok{rename}\NormalTok{ ( }\DataTypeTok{geo =}\NormalTok{ code13 ) }\OperatorTok
\StringTok{                  }\KeywordTok{filter}\NormalTok{ ( }\OperatorTok{!}\KeywordTok{is.na}\NormalTok{(geo)) }\OperatorTok
\StringTok{                  }\KeywordTok{select}\NormalTok{ ( geo, code16, change, resolution ), }
                \DataTypeTok{by =} \StringTok{'geo'}\NormalTok{) }\OperatorTok
\StringTok{    }\KeywordTok{filter}\NormalTok{ ( change }\OperatorTok{!=}\StringTok{ "discontinued"}\NormalTok{) }\OperatorTok
\StringTok{    }\KeywordTok{mutate}\NormalTok{ ( }\DataTypeTok{problem_code =}\NormalTok{ geo ) }\OperatorTok
\StringTok{    }\KeywordTok{mutate}\NormalTok{  ( }\DataTypeTok{geo =}\NormalTok{  code16)}
  
\NormalTok{  recoded_nuts2_}\DecValTok{2013}\NormalTok{ <-}\StringTok{ }\NormalTok{incorrectly_labelled_nuts2_}\DecValTok{2013} \OperatorTok
\StringTok{    }\KeywordTok{filter}\NormalTok{ ( change  }\OperatorTok{==}\StringTok{ "recoded"}\NormalTok{)}
  
\NormalTok{  found_nuts2 <-}\StringTok{ }\NormalTok{recoded_nuts2_}\DecValTok{2013} \OperatorTok
\StringTok{    }\KeywordTok{filter}\NormalTok{ (  geo }\OperatorTok\StringTok{ }\NormalTok{missing_nuts2_}\DecValTok{2016}\NormalTok{  )}
  
  \KeywordTok{message}\NormalTok{ ( }\StringTok{"Found "}\NormalTok{, }\KeywordTok{length}\NormalTok{(}\KeywordTok{unique}\NormalTok{(found_nuts2}\OperatorTok{$}\NormalTok{geo)), }\StringTok{" NUTS2 regions"}\NormalTok{)}
  
  \CommentTok{## You can here pause for a minute.  In some cases, you may want to use }
  \CommentTok{## your NUTS1 level data imputing the missing NUTS2 data. This is a}
  \CommentTok{## good idea if you are not working with atomic count or financial }
  \CommentTok{## information, but relational data like computers per capita.}
  
  
  \CommentTok{## Now let us join the safe observations and see if we can do }
  \CommentTok{## actual imputation. -------------------------------------------}
  \ControlFlowTok{if}\NormalTok{ ( }\KeywordTok{length}\NormalTok{(}\KeywordTok{unique}\NormalTok{(found_nuts2}\OperatorTok{$}\NormalTok{geo)) }\OperatorTok{+}\StringTok{ }\KeywordTok{length}\NormalTok{(}\KeywordTok{unique}\NormalTok{(found_nuts1}\OperatorTok{$}\NormalTok{geo)) }\OperatorTok{==}\StringTok{ }\DecValTok{0}\NormalTok{) \{}
    \KeywordTok{message}\NormalTok{ ( }\StringTok{"There is no data found that can be further arranged. Data is returned in its original format"}\NormalTok{)}
    \KeywordTok{return}\NormalTok{ ( dat )}
\NormalTok{  \}}
  
  
  
\NormalTok{  join_by <-}\StringTok{ }\KeywordTok{names}\NormalTok{ ( correctly_labelled_unchanged ) }
\NormalTok{  join_by <-}\StringTok{ }\NormalTok{join_by [}\KeywordTok{which}\NormalTok{ ( join_by }\OperatorTok\StringTok{ }\KeywordTok{names}\NormalTok{(correctly_labelled_changed) )]}
  
\NormalTok{  join_by2 <-}\StringTok{  }\KeywordTok{names}\NormalTok{ ( correctly_labelled_unchanged )}
\NormalTok{  join_by2 <-}\StringTok{ }\NormalTok{join_by2 [}\KeywordTok{which}\NormalTok{ ( join_by2 }\OperatorTok\StringTok{ }\KeywordTok{names}\NormalTok{(found_nuts1))]}

\NormalTok{  so_far_joined <-}\StringTok{ }\KeywordTok{full_join}\NormalTok{ ( correctly_labelled_unchanged, }
\NormalTok{                        correctly_labelled_changed, }
                        \DataTypeTok{by =}\NormalTok{ join_by ) }\OperatorTok
\StringTok{    }\KeywordTok{full_join}\NormalTok{ ( found_nuts1, }\DataTypeTok{by =}\NormalTok{ join_by2 ) }
  
  \CommentTok{##The number of columns will increase}
\NormalTok{  join_by3 <-}\StringTok{  }\KeywordTok{names}\NormalTok{ ( so_far_joined )}
\NormalTok{  join_by3 <-}\StringTok{ }\NormalTok{join_by3 [}\KeywordTok{which}\NormalTok{ ( join_by3 }\OperatorTok\StringTok{ }\KeywordTok{names}\NormalTok{(found_nuts2))]}
  
\NormalTok{  so_far_joined <-}\StringTok{ }\NormalTok{so_far_joined  }\OperatorTok
\StringTok{    }\KeywordTok{full_join}\NormalTok{ ( found_nuts2, }\DataTypeTok{by =}\NormalTok{ join_by3 )}
  

  \CommentTok{## We have correspondence data only for (most of) the EU countries.}
\NormalTok{  remaining_eu_data <-}\StringTok{ }\NormalTok{tmp }\OperatorTok
\StringTok{    }\KeywordTok{filter}\NormalTok{ ( }\OperatorTok{!}\StringTok{ }\NormalTok{geo }\OperatorTok\StringTok{ }\NormalTok{so_far_joined}\OperatorTok{$}\NormalTok{geo) }
  
  \CommentTok{## These are the known entities that may be used in corrections.}
  \CommentTok{## Make sure that you have at least an empty row for them.}
  
\NormalTok{  used_in_correction <-}\StringTok{ }\KeywordTok{c}\NormalTok{(}\StringTok{"FR24"}\NormalTok{,}
                          \StringTok{"FR26"}\NormalTok{,}\StringTok{"FR43"}\NormalTok{,}
                          \StringTok{"FR23"}\NormalTok{,}\StringTok{"FR25"}\NormalTok{,}
                          \StringTok{"FR22"}\NormalTok{,}\StringTok{"FR30"}\NormalTok{,}
                          \StringTok{"FR21"}\NormalTok{,}\StringTok{"FR41"}\NormalTok{,}\StringTok{"FR42"}\NormalTok{,}
                          \StringTok{"FR51"}\NormalTok{,}
                          \StringTok{"FR52"}\NormalTok{,}
                          \StringTok{"FR53"}\NormalTok{, }\StringTok{"FR61"}\NormalTok{, }\StringTok{"FR63"}\NormalTok{,}
                          \StringTok{"FR62"}\NormalTok{, }\StringTok{"FR81"}\NormalTok{,}
                          \StringTok{"FR7"}\NormalTok{,}
                          \StringTok{"FR82"}\NormalTok{,}
                          \StringTok{"FR83"}\NormalTok{,}
                          \StringTok{"FRA"}\NormalTok{, }
                          \StringTok{"PL11"}\NormalTok{,}\StringTok{"PL33"}\NormalTok{,}
                          \StringTok{"PL3"}\NormalTok{, }
                          \StringTok{"PL12"}\NormalTok{, }
                          \StringTok{"IE023"}\NormalTok{, }\StringTok{"IE024"}\NormalTok{, }\StringTok{"IE025"}\NormalTok{, }
                          \StringTok{"LT00"}\NormalTok{, }\StringTok{"LT00A"}\NormalTok{, }
                          \StringTok{"UKM2"}\NormalTok{,  }
                          \StringTok{"UKM31"}\NormalTok{, }\StringTok{"UKM34"}\NormalTok{, }\StringTok{"UKM35"}\NormalTok{, }\StringTok{"UKM36"}\NormalTok{,}
                          \StringTok{"UKM24"}\NormalTok{, }\StringTok{"UKM32"}\NormalTok{, }\StringTok{"UKM33"}\NormalTok{, }\StringTok{"UKM37"}\NormalTok{, }\StringTok{"UKM38"}\NormalTok{, }
                          \StringTok{"HU102"}\NormalTok{, }\StringTok{"HU101"}
\NormalTok{  )}
  
  \CommentTok{## This is the candidate subset for imputation----------------------}
\NormalTok{  correct_with_correspondence <-}\StringTok{ }\NormalTok{remaining_eu_data }\OperatorTok
\StringTok{    }\KeywordTok{select}\NormalTok{ ( time, country_code, years, indicator, geo, values ) }
  
  \CommentTok{## The data has time and space dimension, so corrections have to be }
  \CommentTok{## made for each year in the dataset, especially because usually }
  \CommentTok{## the labelling is not consistent through the years.}
  
\NormalTok{  correspondence_by_year <-}\StringTok{ }\ControlFlowTok{function}\NormalTok{ (df, this_time ) \{}
\NormalTok{    df <-}\StringTok{ }\NormalTok{df }\OperatorTok\StringTok{ }\KeywordTok{filter}\NormalTok{ ( time }\OperatorTok{==}\StringTok{ }\NormalTok{this_time )}
    
    \CommentTok{##To make joining possible, you must have at least empty rows}
    \CommentTok{##for the changed regions.}
    \CommentTok{##Do not forget that the Eurostat codes are not ISO-conform in}
    \CommentTok{##the case of the United Kingdom and Greece. }
    \CommentTok{##This applies to the regional codes, too. }
\NormalTok{    complete_with_missing <-}\StringTok{ }\KeywordTok{tibble}\NormalTok{ (}
      \DataTypeTok{geo  =}\NormalTok{ used_in_correction[}\KeywordTok{which}\NormalTok{ ( }\OperatorTok{!}\StringTok{ }\KeywordTok{unique}\NormalTok{(used_in_correction) }\OperatorTok\StringTok{ }\NormalTok{df}\OperatorTok{$}\NormalTok{geo)], }
      \DataTypeTok{country_code =} \KeywordTok{case_when}\NormalTok{ ( }
        \KeywordTok{substr}\NormalTok{(geo,}\DecValTok{1}\NormalTok{,}\DecValTok{2}\NormalTok{) }\OperatorTok{==}\StringTok{ "UK"} \OperatorTok{~}\StringTok{ "GB"}\NormalTok{,}
        \KeywordTok{substr}\NormalTok{(geo,}\DecValTok{1}\NormalTok{,}\DecValTok{2}\NormalTok{) }\OperatorTok{==}\StringTok{ "EL"} \OperatorTok{~}\StringTok{ "GR"}\NormalTok{,}
        \OtherTok{TRUE} \OperatorTok{~}\StringTok{ }\KeywordTok{substr}\NormalTok{(geo,}\DecValTok{1}\NormalTok{,}\DecValTok{2}\NormalTok{)), }
      \DataTypeTok{values =} \OtherTok{NA_real_}\NormalTok{, }
      \DataTypeTok{indicator  =} \OtherTok{NA_character_}\NormalTok{, }
      \DataTypeTok{time =}\NormalTok{ this_time, }
      \DataTypeTok{years =} \KeywordTok{as.numeric}\NormalTok{(}\KeywordTok{substr}\NormalTok{(}\KeywordTok{as.character}\NormalTok{ (this_time),}\DecValTok{1}\NormalTok{,}\DecValTok{4}\NormalTok{))}
\NormalTok{    )}
    
    \CommentTok{##Try all possible improvements, though most of this will be missing}
    \CommentTok{##in most datasets. }
\NormalTok{    correct_with_correspondence <-}\StringTok{ }\NormalTok{df }\OperatorTok
\StringTok{      }\KeywordTok{full_join}\NormalTok{ (., complete_with_missing, }
                 \DataTypeTok{by =} \KeywordTok{c}\NormalTok{(}\StringTok{"indicator"}\NormalTok{, }\StringTok{"geo"}\NormalTok{, }\StringTok{"values"}\NormalTok{, }\StringTok{"years"}\NormalTok{,}
                        \StringTok{"time"}\NormalTok{, }\StringTok{"country_code"}\NormalTok{) ) }\OperatorTok
\StringTok{      }\KeywordTok{fill}\NormalTok{ ( indicator )  }\OperatorTok
\StringTok{      }\KeywordTok{spread}\NormalTok{ ( geo, values  ) }\OperatorTok
\StringTok{      }\KeywordTok{mutate}\NormalTok{ ( }\DataTypeTok{FRB=}\NormalTok{FR24,}
               \DataTypeTok{FRC=}\NormalTok{FR26}\OperatorTok{+}\NormalTok{FR43,}
               \DataTypeTok{FRD=}\NormalTok{FR23}\OperatorTok{+}\NormalTok{FR25,}
               \DataTypeTok{FRE=}\NormalTok{FR22}\OperatorTok{+}\NormalTok{FR30,}
               \DataTypeTok{FRF=}\NormalTok{FR21}\OperatorTok{+}\NormalTok{FR41}\OperatorTok{+}\NormalTok{FR42,}
               \DataTypeTok{FRG=}\NormalTok{FR51,}
               \DataTypeTok{FRH=}\NormalTok{FR52,}
               \DataTypeTok{FRI=}\NormalTok{FR53}\OperatorTok{+}\NormalTok{FR61}\OperatorTok{+}\NormalTok{FR63,}
               \DataTypeTok{FRJ=}\NormalTok{FR62}\OperatorTok{+}\NormalTok{FR81,}
               \DataTypeTok{FRK=}\NormalTok{FR7,}
               \DataTypeTok{FRL=}\NormalTok{FR82,}
               \DataTypeTok{FRM=}\NormalTok{FR83,}
               \DataTypeTok{FRY=}\NormalTok{FRA, }
               \DataTypeTok{LT02=}\NormalTok{LT00}\OperatorTok{-}\NormalTok{LT00A,}
               \DataTypeTok{UKM7=}\NormalTok{UKM2}\OperatorTok{-}\NormalTok{UKM24,}
               \DataTypeTok{UKM8=}\NormalTok{UKM31}\OperatorTok{+}\NormalTok{UKM34}\OperatorTok{+}\NormalTok{UKM35}\OperatorTok{+}\NormalTok{UKM36,}
               \DataTypeTok{UKM9=}\NormalTok{UKM24}\OperatorTok{+}\NormalTok{UKM32}\OperatorTok{+}\NormalTok{UKM33}\OperatorTok{+}\NormalTok{UKM37}\OperatorTok{+}\NormalTok{UKM38, }
               \DataTypeTok{PL7=}\NormalTok{PL11}\OperatorTok{+}\NormalTok{PL33,}
               \DataTypeTok{PL8=}\NormalTok{PL3}\OperatorTok{-}\NormalTok{PL33,}
               \DataTypeTok{PL9=}\NormalTok{PL12,}
               \DataTypeTok{IE05=}\NormalTok{IE023}\OperatorTok{+}\NormalTok{IE024}\OperatorTok{+}\NormalTok{IE025,}
               \DataTypeTok{HU11=}\NormalTok{HU101,}
               \DataTypeTok{HU12=}\NormalTok{HU102) }\OperatorTok
\StringTok{      }\KeywordTok{gather}\NormalTok{ ( geo, values, }\OperatorTok{-}\KeywordTok{one_of}\NormalTok{(}\StringTok{"indicator"}\NormalTok{, }\StringTok{"country_code"}\NormalTok{, }\StringTok{"time"}\NormalTok{, }\StringTok{"years"}\NormalTok{)) }\OperatorTok
\StringTok{      }\KeywordTok{filter}\NormalTok{ ( }\OperatorTok{!}\KeywordTok{is.na}\NormalTok{(values)) }\OperatorTok
\StringTok{      }\KeywordTok{filter}\NormalTok{ ( geo }\OperatorTok\StringTok{ }\KeywordTok{c}\NormalTok{(missing_nuts1_}\DecValTok{2016}\NormalTok{, missing_nuts2_}\DecValTok{2016}\NormalTok{ ))}
    
    \CommentTok{#Return the long form table for the given year.}
\NormalTok{    correct_with_correspondence}
\NormalTok{  \}}
  
  \ControlFlowTok{for}\NormalTok{ ( i }\ControlFlowTok{in} \KeywordTok{seq_along}\NormalTok{(}\KeywordTok{unique}\NormalTok{(correct_with_correspondence}\OperatorTok{$}\NormalTok{time))) \{}
\NormalTok{    this_time <-}\StringTok{ }\KeywordTok{unique}\NormalTok{(correct_with_correspondence}\OperatorTok{$}\NormalTok{time)[i]}
    \ControlFlowTok{if}\NormalTok{ ( i }\OperatorTok{==}\StringTok{ }\DecValTok{1}\NormalTok{ ) \{}
\NormalTok{      corrected_with_correspondence <-}\StringTok{ }\KeywordTok{correspondence_by_year}\NormalTok{ (correct_with_correspondence, }
                                                               \DataTypeTok{this_time =}\NormalTok{ this_time ) }
\NormalTok{    \} }\ControlFlowTok{else}\NormalTok{ \{}
\NormalTok{      tmp <-}\StringTok{ }\KeywordTok{correspondence_by_year}\NormalTok{ (correct_with_correspondence, }
                                     \DataTypeTok{this_time =}\NormalTok{ this_time ) }
      \ControlFlowTok{if}\NormalTok{ ( }\KeywordTok{is.null}\NormalTok{(tmp)) }\ControlFlowTok{next} 
      \ControlFlowTok{if}\NormalTok{ ( }\OperatorTok{!}\KeywordTok{nrow}\NormalTok{(tmp)}\OperatorTok{==}\DecValTok{0}\NormalTok{ ) \{}
\NormalTok{        corrected_with_correspondence <-}\StringTok{ }\KeywordTok{full_join}\NormalTok{ (}
\NormalTok{          corrected_with_correspondence ,tmp,}
          \DataTypeTok{by =} \KeywordTok{c}\NormalTok{(}\StringTok{"time"}\NormalTok{, }\StringTok{"country_code"}\NormalTok{, }\StringTok{"years"}\NormalTok{, }\StringTok{"indicator"}\NormalTok{,}
                 \StringTok{"geo"}\NormalTok{, }\StringTok{"values"}\NormalTok{)) }
\NormalTok{      \} }
      
\NormalTok{    \} }\CommentTok{#end of else}
\NormalTok{  \} }\CommentTok{#end of loop}
  
  \CommentTok{## There are cases where the correspondence table is silent, or}
  \CommentTok{## I have not found a solution yet. }
  \CommentTok{## Let us give a warning about them.}
  
\NormalTok{  incorrectly_labelled_unknown <-}\StringTok{ }\NormalTok{tmp3 }\OperatorTok
\StringTok{    }\KeywordTok{filter}\NormalTok{ ( }\OperatorTok{!}\StringTok{ }\NormalTok{geo }\OperatorTok\StringTok{ }\KeywordTok{c}\NormalTok{(changed_regions}\OperatorTok{$}\NormalTok{code16, changed_regions}\OperatorTok{$}\NormalTok{code13))}
  
  \ControlFlowTok{if}\NormalTok{ ( }\KeywordTok{nrow}\NormalTok{(incorrectly_labelled_unknown ) }\OperatorTok{>}\StringTok{ }\DecValTok{0}\NormalTok{ ) \{}
      \KeywordTok{message}\NormalTok{ ( }\StringTok{"The following labels do not conform the NUTS2016 definition: "}\NormalTok{, }\KeywordTok{paste}\NormalTok{(}
      \KeywordTok{unique}\NormalTok{(incorrectly_labelled_unknown}\OperatorTok{$}\NormalTok{geo), }\DataTypeTok{collapse =} \StringTok{","}\NormalTok{ )}
\NormalTok{    )}
    \KeywordTok{warning}\NormalTok{ (}\StringTok{"Unknown labels found"}\NormalTok{)}
\NormalTok{  \} }
  
  \CommentTok{##Now you may have identical data with NUTS2013 and NUTS2016 geocodes}
  \CommentTok{##if the labels are the same.}
\NormalTok{  corrected_dataset <-}\StringTok{ }\KeywordTok{full_join}\NormalTok{ ( }
\NormalTok{    so_far_joined,  corrected_with_correspondence, }
    \DataTypeTok{by =} \KeywordTok{c}\NormalTok{(}\StringTok{"geo"}\NormalTok{, }\StringTok{"values"}\NormalTok{, }\StringTok{"indicator"}\NormalTok{, }\StringTok{'years'}\NormalTok{, }
           \StringTok{'time'}\NormalTok{, }\StringTok{'country_code'}\NormalTok{) ) }\OperatorTok
\StringTok{    }\KeywordTok{fill}\NormalTok{ ( unit ) }\OperatorTok
\StringTok{    }\KeywordTok{add_count}\NormalTok{ ( geo, time )}
  
  \ControlFlowTok{if}\NormalTok{ ( }\KeywordTok{any}\NormalTok{(corrected_dataset}\OperatorTok{$}\NormalTok{n}\OperatorTok{>}\DecValTok{1}\NormalTok{)) \{}
    \KeywordTok{message}\NormalTok{(}\StringTok{"Duplications occured, probably because of the relabelling."}\NormalTok{)}
\NormalTok{    \}}
  
  \CommentTok{## Hopefully that is the only reason of the duplication, which }
  \CommentTok{## is the case if not only the time and geo variable but the }
  \CommentTok{## actual value of the statistical indicator matches.}
\NormalTok{  corrected_dataset <-}\StringTok{ }\NormalTok{corrected_dataset }\OperatorTok\StringTok{ }\KeywordTok{select}\NormalTok{ (}\OperatorTok{-}\NormalTok{n) }\OperatorTok
\StringTok{    }\KeywordTok{ungroup}\NormalTok{() }\OperatorTok
\StringTok{    }\KeywordTok{distinct}\NormalTok{ ( geo, time, values, }\DataTypeTok{.keep_all =} \OtherTok{TRUE}\NormalTok{) }\OperatorTok
\StringTok{    }\KeywordTok{add_count}\NormalTok{ ( geo, time )}
  
  \ControlFlowTok{if}\NormalTok{ ( }\KeywordTok{any}\NormalTok{(corrected_dataset}\OperatorTok{$}\NormalTok{n}\OperatorTok{>}\DecValTok{1}\NormalTok{)) \{}
    \CommentTok{# You can add here further exception handling code if you want.}
    \KeywordTok{warning}\NormalTok{(}\StringTok{"The duplications could not be resolved fully, please review manually"}\NormalTok{)}
\NormalTok{  \} }\ControlFlowTok{else}\NormalTok{ \{}
   \KeywordTok{message}\NormalTok{ (}\StringTok{"The duplications were resolved successfully."}\NormalTok{)  }
\NormalTok{  \}}
  
  \CommentTok{##To document your work, here is the log file for this regional}
  \CommentTok{##statistical product.}
  \KeywordTok{message}\NormalTok{ ( }\StringTok{"Corrections: "}\NormalTok{, }\KeywordTok{paste}\NormalTok{(}
    \KeywordTok{c}\NormalTok{(missing_nuts1_}\DecValTok{2016}\NormalTok{, missing_nuts2_}\DecValTok{2016}\NormalTok{ ) [}\KeywordTok{c}\NormalTok{(missing_nuts1_}\DecValTok{2016}\NormalTok{, missing_nuts2_}\DecValTok{2016}\NormalTok{ ) }\OperatorTok\StringTok{  }\NormalTok{corrected_dataset}\OperatorTok{$}\NormalTok{geo], }
    \DataTypeTok{collapse =} \StringTok{", "}
\NormalTok{  ))}
  
  \CommentTok{## And now return the results:}
\NormalTok{  corrected_dataset}
\NormalTok{\}}
\end{Highlighting}
\end{Shaded}

\hypertarget{conclusions}{%
\subsection{Conclusions}\label{conclusions}}

While I did write a comprehensive code to these problems, it has so many
excemptions that I do not suggest do include it in the \texttt{eurostat}
package, because it would require a lot of work to test all possible
errors and to maintain the quality of the code as further changes may
come in the future.

\hypertarget{simple-filtering-and-error-handling}{%
\subsubsection{Simple filtering and error
handling}\label{simple-filtering-and-error-handling}}

I think that the simple cases can and should be encoded in the eurostat
package. For example, a very simple function could create two additional
columns from the \texttt{geo} variable, like I did, a \texttt{code13}
and a \texttt{code16} one, which contains the correct code for
\texttt{NUTS2013} for backward compatiblity with pre-exsiting maps and
reseaerch, and the current \texttt{NUTS2016} definition. The
\texttt{eurostat} package should also allow the use of \texttt{NUTS2013}
and \texttt{NUTS2016} maps.

I also believe that a very simple fix could be done to create, whenever
they are missing, the obviously equivalent rows. For example, if there
is a \texttt{LU} or \texttt{LU0} data present, it should be mutated into
a \texttt{LU00} row, too. This is not prone to error, and allows the use
of \texttt{NUTS0}, \texttt{NUTS1} and \texttt{NUTS2} maps joining
datasets.

\hypertarget{warnings}{%
\subsubsection{Warnings}\label{warnings}}

In the case of \texttt{discontinued} and \texttt{changed} \texttt{geo}
definitions, I think that the \texttt{get\_eurostat} function should
give a warning, someling like ``Beware, the downloaded data contain geo
codes that follow the earlier national and regional coding nomenclature.
Read our vignette.''

\hypertarget{consultation-with-eurostat}{%
\subsubsection{Consultation with
Eurostat}\label{consultation-with-eurostat}}

There is a statistical regulation on how the NUTS change have to be
carried out in statistics, and I am sure that Eurostat more or less
complies with them, but it does not mean that it complies with the best
possible way. The current solution is extremely error-prone and it
should be recommended to Eurostat to change it.

\begin{itemize}
\item
  The inconsistent label levels and label codes make the joining of
  different statistical tables almost impossible. Especially French,
  German, British and Polish data, furthermore the small country data is
  almost always lost in joining operations. It would be far better to
  provide table with two coding columns, like the solution above with
  \texttt{geo\_code13} and \texttt{geo\_code16}, or with the addition of
  an explicit auxillary column saying that the actual row is defined as
  \texttt{NUTS2013} or \texttt{NUTS2016}. This is not unambigous,
  because in some cases the codes changed, and in other they did not. A
  \texttt{PL2} code can be both \texttt{NUTS2013} or \texttt{NUTS2016}.
\item
  The mixed presence of \texttt{NUTS2013} and \texttt{NUTS2016} data
  makes panel data arrangement next to impossible. These should be
  separate statistical products with separate name and metadata
  explanations.
\item
  Some products go back as far as the \texttt{NUTS2003} definitions, and
  it should be clear, and if possible, consistent, how transition from
  \texttt{NUTS2013} to \texttt{NUTS2006}, then to \texttt{NUTS2010},
  then to \texttt{NUTS2013} and eventually \texttt{NUTS2016} happened.
\item
  The introduction of mixed level statistical products, like
  \href{https://appsso.eurostat.ec.europa.eu/nui/show.do?dataset=isoc_r_iuse_i\&lang=en}{isoc\_r\_iuse\_i},
  is welcome, because it allows users to follow the correspondence
  tables, but it should be done with correct descriptions and in a
  consistent way.
\item
  It is a bad practice that whenever NUTS or NACE or COFOG or other
  statistical definitions change, Eurostat withdraws statistical
  products. These products should be made available forever in an
  archive section, with warnings that they are not subject to further
  revisions and improvements, and they use out-of-date statistical
  definitions. Such datasets may be used in many academic and
  non-academic populations, which cannot be reviewed and fact checked if
  the data disappears. With the increasing use of reproducible research
  techniques in academia and among professional users, disappearing
  datasets, or datasets that are not renamed but changed in essence
  cause extraordinary amont of debugging work.
\end{itemize}


\end{document}
